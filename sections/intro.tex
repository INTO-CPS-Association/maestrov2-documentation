\section{Introduction}\label{sec:intro}

Maestro is a framework built for orchestrating co-simulations based on the
Functional Mock-Up Interface 2.0 standard for co-simulation.

TODO: Insert figure with: Program, Verification, Runtime to present the
high-level idea.

The framework is divided into two main parts: Maestro-Program and
Maestro-Runtime. These entities and the flow of conducting a co-simulation using
MaestroV2 is depicted in \cref{fig:conducting_co-simulation-overview} and
described in this section.

\begin{description}
  \item[Maestro-Program] is the entity that controls the process of creating a
program. A Program specifies how to conduct a co-simulation and consists of
commands to be carried out by Maestro-Runtime. In order to create a Program,
Maestro-Program employs plugins. The creation of a Program is referred to as the
Program Phase.
  \item[Maestro-Runtime] is the entity that controls the process of executing a
program. The execution of a program is referred to as the Execution Phase.

\end{description}
\begin{figure}[htb] \centering
\includegraphics[width=\textwidth]{figures/conducting_co-simulation_overview.pdf}
  \caption{Conducting a co-simulation with Maestro-Program and Maestro-Runtime}
  \label{fig:conducting_co-simulation-overview}
\end{figure}

The following paragraphs describes the content of the figure. Initially, some
terminology and definitions are presented followed by a description of the
consecutive behavior.
\begin{description}
  \item[Context] Context is data and information related to the co-simulation.
    For example, the FMUs to employ in a given co-simulation or the dependencies
    between the variables of the FMUs for a given co-simulation.
  \item[Environment] The environment is information related to variables. For
    example, type information or the value of a given variable.
  \item[ProgramContext] Terminology for the Context used in relation to
    Maestro-Program.
  \item[ProgramEnvironment] Terminology for the Environment used in relation to
    Maestro-Program.
  \item[RuntimeContext] Terminology for Context used in relation to
    Maestro-Runtime.
  \item[RuntimeEnvironment] Terminology for the Environment used in relation to
    Maestro-Runtime. This contains i.e. variables in scope and values of variables.
  \item[PluginContext] A plugin specific Context. Maestro-Runtime does not know
    its content and the code to process it must be provided by the respective
    plugin.
    \item[Configuration] Configuration describing how to create the Program,
    i.e. which FMUs and plugins to use.
  \item[Root Context] Terminology for the initial Program Context. Examples of
    data in the Root Context is i.e. FMUs to use in a
    co-simulation and parameters for the FMUs.
  \item[Program] A Program is a complete in the sense that it can be passed to
    Maestro-Runtime for execution as opposed to a ProgramFragment described below.
  \item[Program Fragment] A Program Fragment is part of a Program.
  \item[Plugin] During the Program phase a plugin can read/update the Program
    Fragment, and/or read/update the ProgramContext and/or read/update the
    ProgramEnvironment. An example of ProgramContext information that a plugin can
    add is the dependencies between the variables of the FMUs. An example of a
    Program Fragment that a plugin can create is the necessary commands to perform
    initialisation of the FMUs. During the Runtime phase a plugin can read/update
    the RuntimeEnvironment, read/update the PluginContext and/or create commands to
    be executed immediately.
\end{description}

The Activator in \cref{fig:conducting_co-simulation-overview} is the entity
(person or tool) that launches a co-simulation. The Activator shall provide a
Configuration. See TODO.

Maestro-Program invokes the plugins according to the configuration. This
is demonstrated in \cref{fig:conducting_co-simulation-overview} where Plugin 1
receives ProgramContext 1, ProgramEnvironment 1, ProgramFragment 1 and
creates: ProgramContext 2, ProgramEnvironment 2 and ProgramFragment 2. This is
then passed to Plugin 2 and so on until no more plugins are specified. At this
stage it is expected that a Program has been created. It is then possible to
verify the Program, which is also based on plugins. The verification plugins can
report their results but cannot update the Program, context or environment. The
result of the Program phase is:
A Program (Program in the figure), a Context (Context X in the figure) and an Environment (Environment X
in the figure)

Maestro-Runtime executes the Program and can utilise the related Context and Environment.
It is possibly to create commands in
the Program that prompts Maestro-Runtime to invoke a specific plugin. The plugin
will be invoked with the initial Context (Context X in the figure), the Runtime
Environment (RunTimeEnvironment in the figure) and a Context for the specific plugin (PluginContext in the figure).



\subsection{TO BE DONE}
\begin{itemize}

\end{itemize}

%%% Local Variables: %%% mode: latex %%% TeX-master: "../Maestro" %%% End:
