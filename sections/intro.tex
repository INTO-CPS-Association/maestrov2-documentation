\section{Introduction}\label{sec:intro}
Maestro is a plugin-based framework built for specifying and executing co-simulations based
on The Functional Mock-up Interface 2.0 and 2.1 standard for co-simulation. It
is currently being developed.

There are three steps involved in conducting a co-simulation using Maestro:
\begin{enumerate}
  \item Creating a specification of a FMI-based co-simulation to be executed.
  \item Verifying the specification.
  \item Executing the specification.
\end{enumerate}

The specification of a co-simulation is based on a Domain Specific Language (DSL) called Maestro
Base Language (MaBL). An MaBL specification is created by adding, editing and/or
removing nodes in the Abstract Syntax Tree (AST) of the specification in question and not by
writing it by hand. The main
reason for this approach is the effort involved in developing Integrated Development
Environment functionality for MaBL. Some of the major goals for this language are:
\begin{description}
  \item[Goal 1:] MaBL shall be capable of expressing both industry and
    research-related co-simulation scenarios.
  \item[Goal 2:] MaBL shall be constrained such that specifications can be
    verified.
    \item[Goal 3:] It shall be possible to serialise an MaBL specification such
    that it can be passed to a runtime for execution.
\end{description}
It is expected that aligning these goals will present challenges and for this
reason, the initial version of MaBL does not cover all considered features and
use cases. The considered features and use cases are available at TODO REF
FEATURES AND USE CASES.

The process of creating a specification is to be carried out via plugins, which
leads to another part of the Maestro framework. Maestro is based on plugins that
offer certain functionality. For example, a plugin can generate the MaBL
specification for initializing the FMUs of a co-simulation. Plugins can be
combined such that the joined MaBL specification specifies a complete
co-simulation that can be executed by a runtime.

The previous paragraph mentioned ``a complete co-simulation'', which has not
been qualified. This leads us to the next part - verifying the specification.
The verification of specifications, also based on plugins, will give meaning
to ``a complete co-simulation'', which might be different from scenario to scenario.

The final part of conducting a co-simulation is to execute the verified
specification. Maestro is envisioned to provide a C++-based runtime for
executing MaBL specifications. However it is possible to provide other runtimes since a MaBL specification can be
serialised. For the reason, creating, verifying and executing specifications are
decoupled.

The features and use cases being considered to provide support for by Maestro is
described in \cref{sec:features_use-cases}. \cref{sec:approach} presents an
extended description of how the different components of Maestro are joined to
conduct a co-simulation. Next, \cref{sec:mabl} presents the MaBL specification
language. The publication then elaborates on plugins for creating and verifying
specifications in \cref{sec:plugins}. Afterwards, \cref{sec:examples} presents
examples and finally, \cref{sec:future-work} presents ideas
on the future work of Maestro.


%%% Local Variables:
%%% mode: latex
%%% TeX-master: "../Maestro"
%%% End:
