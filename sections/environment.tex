\section{Environment}
The Environment contains information available to Maestro-Program,
plugins and Maestro-Runtime. These entities can also add information to the
Environment by creating a new Environment, which shall contain data from the old
environment that has not been updated.

\subsection{Environment Structure}
This section describes the structure of the Root Environment and thereby what is
supported natively. The reason for natively supporting some data is that it is
considered essential for FMI co-simulation.
First, the entries are described in an overall format. Afterwards, the types
mentioned in the overall description. Note, that not all entries are populated
by Maestro-Program prior to employing plugins. Some of these entries will be
populated by plugins. TODO: Clearly describe which entries Maestro-Program
populates.
\paragraph{Overall Structure}
\begin{description}
  \item[RawFMUs] The raw FMU information. Perhaps from a UI. Type: Map[FmuKey, FmuPath]
  \item[RawInstances] The raw instance information. Perhaps from a UI. Type: Map[FmuKey, List[InstanceKey]]
  \item[RawConnections] The raw connections information. Perhaps from a UI.
    Type: Map[{FmuKey, FmuInstance, ScalarVariableName}, Set[{FmuKey, FmuInstance, ScalarVariableName}]]
  \item[RawParameters] The raw parameters information. Perhaps from a UI.
    Type: Map[ParameterKey, Value] and Map[FMU, Map[Instance, Map[ScalarVariableName, ParameterKey]]]
  \item[FMUsWithInstances] Enriches RawFMUs with ModelDescription information
    and connects FMUs to their respective instances. Type: Map[Fmu, Set[Instance]]
  \item[Connections] The connections based on FMUsWithInstances and variables
    from the corresponding ModelDescription files. Type: Set[Connection]
  \item[SortedDependantVariables] An list that describes the order of setting
    and getting dependant scalar variables according to internal and external
    dependencies. Type: List[ScalarVariableID]
    \item[Custom] This entry can be used freely. Type: Map[CustomDataKey, Any].
\end{description}

\paragraph{Types}
\begin{description}
  \item[FmuKey] String: Unique identifier for a given FMU
  \item[FmuPath] URI: Location of the FMU
  \item[InstanceKey] String: Unique identifier for a given instance
  \item[ScalarVariableName] String: Name of a given scalar variable.
    TODO: Perhaps valuereference, perhaps both?
  \item[Fmu] Data Object with FmuKey, FmuPath and ModelDescription.
  \item[Instance] Data Object that consists of parent FMU and InstanceKey
  \item[ModelDescription] The parsed model description.
  \item[Connection] Data Object with from of type ScalarVariableID and to of type Set[ScalarVariableID]
  \item[ScalarVariableID] Data Object that consists of FmuKey,
    InstanceKey, ScalarVariableValueReference
  \item[CustomDataKey] String: Unique identifier for some custom data.
  \item[Any] Represents the value associated with a CustomDataKey.
    TODO: Represent as Sum Type: Any (Any) | JSON (String) | Text (String) |
    Byte (Array[Byte]) | ?
\end{description}

%%% Local Variables:
%%% mode: latex
%%% TeX-master: "../Maestro"
%%% End:
