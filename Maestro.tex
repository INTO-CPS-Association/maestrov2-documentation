% !TeX encoding = UTF-8
% !TeX spellcheck = en_GB

%%% Add [final] option to the report class to switch between draft and final version of the report
%%% Use [narrowmargin] to enable narrow margins - this may impair readability.
%\documentclass[a4paper,12pt,draft]{include/intocpsreport}   %Or
\documentclass[a4paper,12pt,final]{include/intocpsassociation}   %Or
% intocpslargereport if chapters are required.
%
%
%
\usepackage[T1]{fontenc}
\usepackage[utf8]{inputenc}
\usepackage{longtable}
\usepackage{tikz-uml}
\usepackage{framed}
\usepackage{subcaption}
\usepackage[hyphenbreaks]{breakurl}
\usepackage{color}
\usepackage{amsmath}
\usepackage{courier}
\usepackage{xspace}
\usepackage{cleveref}
\usepackage{subcaption}
\usepackage{textcomp} % Used for 20-sim section \textrightarrow
%\usepackage{showframe}

\usepackage{listings}
\usepackage{glossaries}

%% Define listing environment for XML
\definecolor{gray}{rgb}{0.4,0.4,0.4}
\definecolor{darkblue}{rgb}{0.0,0.0,0.6}
\definecolor{cyan}{rgb}{0.0,0.6,0.6}

\lstset{
  basicstyle=\footnotesize\ttfamily,
  columns=fullflexible,
  showstringspaces=false,
  commentstyle=\color{gray}\upshape
}

\lstdefinelanguage{XML}
{
  morestring=[b]",
  morestring=[s]{>}{<},
  morecomment=[s]{<?}{?>},
  stringstyle=\color{black},
  identifierstyle=\color{darkblue},
  keywordstyle=\color{cyan},
  morekeywords={xmlns,version,type}% list your attributes here
}


\lstnewenvironment{xml}[1][]{\lstset{  language=XML,
  morekeywords={encoding, xs:schema,xs:element,xs:complexType,xs:sequence,xs:attribute}}\lstset{#1}}
{}

%% end listing environment for XML
%
%
%
\def\draftnote#1{\noindent\smallskip\framebox{\begin{minipage}{0.95\columnwidth}\color{red}#1\end{minipage}}\smallskip\par}
\newenvironment{draftnoteenv}{\noindent\smallskip\begin{framed}\begin{minipage}{0.95\columnwidth}\color{red}}{\end{minipage}\end{framed}\smallskip\par}
\newenvironment{assumption}{\noindent\smallskip\color{blue}\begin{framed}\begin{minipage}{0.95\columnwidth}}{\end{minipage}\end{framed}\smallskip\par}
%
%
%
\newcommand{\revisit}[1]{\textcolor{red}{\pmb{[[[}\@ #1\@ \pmb{]]]}}}
%
%
\reporttitle{INTO-CPS Maestro Documentation}
\shortreporttitle{INTO-CPS Maestro Documentation}  %To use if report title is too long for header
%
%
%
%%% Set document release class as appropriate
%%% e.g. Public, Restricted, Programme Participant
\reportstatus{Public Draft}
%
%
%
%%% If document is a deliverable, this flag should be commented out
%%% e.g. %\technotetrue
%%% If report is a technical report, leave uncommented
%%% e.g. \technotetrue
\technotetrue % Comment out as appropriate
%
%
%
\submissiondate{}
\contributors{
Casper Thule, Aarhus University, Centre for Digital Twins \\
Kenneth Lausdahl, Mj{\o}lner Informatics A/S \\
Cl\'audio Gomes, Aarhus University, Centre for Digital Twins \\
Hugo Daniel Macedo, Aarhus University, Centre for Digital Twins

}
%
%
%
\editors{
Casper Thule, Aarhus University, Centre for Digital Twins \\
}
%
%
%
%\reviewers{Ken Pierce, UNEW\\
%Kangfeng Ye, UY\\
%Luis Diogo Couto, UTRC}
%
%
%
%% Version details
% #1: version
% #2: date
% #3: author
% #4: description
\addversion{0.01}{November 18, 2019}{Casper Thule}{Initial Version.}

%
%
\begin{document}
\maketitle
%
%
%
%%%% Document abstract page %%%%
\section*{Abstract}
\label{sec:abstract}
%
Maestro is a co-simulation orchestration engine for FMI 2.0 originally developed as part of the INTO-CPS project from 2016-2018. This publication
concerns the new Maestro that will take over for the existing tool.

The motivation for recreating Maestro is to make it more flexible and provide
better support for verification. This will be carried out by defining a language
called Maestro Base Language (MaBL) that can be used to create a specification
of a co-simulation to be carried out. The flexibility is introduced by enabling
the possibility to combine the constructs of the language in various ways to
support different scenarios. Having a specification of a co-simulation that can
be passed to verification tools is an improvement in terms of verification, as
it is now clear what is to be executed. MaBL has been constrained in in order to
simplify this process. Finally, Maestro will also incorporate a runtime that can
execute MaBL specifications.

The flexibility is also expressed in the nature of creating a specification and
verifying it, as these phases are based on plugins. Furthermore, plugins can be composed such
that they jointly represent a specification or multiple verification efforts. Thus, if something is
missing in order to support a certain scenario or verification task, it is
possible to create a plugin that Maestro will make use of.


The ambition is that Maestro can support both research-oriented efforts in
co-simulation but also industry-related activities. As the development is in
progress, the Maestro team is very interested in feedback, features, use cases,
plugins and similar. The repository is available at \url{https://github.com/into-cps-association/maestrov2}
\newpage
%
%%%% Document table of contents page %%%%
\tableofcontents
\newpage
%
%
%
%%%% Document Content %%%%
%% \chapter{Chapter Title} %% if intocpslargereport is in use
%\begin{assumption}
%
%
%
\section{Introduction}\label{sec:intro}
Maestro is a plugin-based framework built for specifying and executing co-simulations based
on The Functional Mock-up Interface 2.0 and 2.1 standard for co-simulation. It
is currently being developed.

There are three steps involved in conducting a co-simulation using Maestro:
\begin{enumerate}
  \item Creating a specification of a FMI-based co-simulation to be executed.
  \item Verifying the specification.
  \item Executing the specification.
\end{enumerate}

The specification of a co-simulation is based on a Domain Specific Language (DSL) called Maestro
Base Language (MaBL). An MaBL specification is created by adding, editing and/or
removing nodes in the Abstract Syntax Tree (AST) of the specification in question and not by
writing it by hand. The main
reason for this approach is the effort involved in developing Integrated Development
Environment functionality for MaBL. Some of the major goals for this language are:
\begin{description}
  \item[Goal 1:] MaBL shall be capable of expressing both industry and
    research-related co-simulation scenarios.
  \item[Goal 2:] MaBL shall be constrained and explicit such that specifications can be
    verified.
    \item[Goal 3:] It shall be possible to serialise an MaBL specification such
    that it can be passed to a runtime for execution.
\end{description}
It is expected that aligning these goals will present challenges and for this
reason, the initial version of MaBL does not cover all considered features and
use cases. The considered features and use cases are available at TODO REF
FEATURES AND USE CASES.

The process of creating a specification is to be carried out via plugins, which
leads to another part of the Maestro framework. Maestro is based on plugins that
offer certain functionality. For example, a plugin can generate the MaBL
specification for initializing the FMUs of a co-simulation. Plugins can be
combined such that the joined MaBL specification specifies a complete
co-simulation that can be executed by a runtime.

The previous paragraph mentioned ``a complete co-simulation'', which has not
been qualified. This leads us to the next part - verifying the specification.
The verification of specifications, also based on plugins, will give meaning
to ``a complete co-simulation'', which might be different from scenario to scenario.

The final part of conducting a co-simulation is to execute the verified
specification. Maestro is envisioned to provide a C++-based runtime for
executing MaBL specifications. However it is possible to provide other runtimes since a MaBL specification can be
serialised. For the reason, creating, verifying and executing specifications are
decoupled.

The features and use cases being considered to provide support for by Maestro is
described in \cref{sec:features_use-cases}. \cref{sec:approach} presents an
extended description of how the different components of Maestro are joined to
conduct a co-simulation. Next, \cref{sec:mabl} presents the MaBL specification
language. The publication then elaborates on plugins for creating and verifying
specifications in \cref{sec:plugins}. Afterwards, \cref{sec:examples} presents
examples and finally, \cref{sec:future-work} presents ideas
on the future work of Maestro.


%%% Local Variables:
%%% mode: latex
%%% TeX-master: "../Maestro"
%%% End:

\clearpage
\section{Features and Use Cases}\label{sec:features_use-cases}
Several features and use cases has been identified in order to get an overview
of the co-simulation domain that Maestro is part of. The starting point for this
work is a survey of co-simulation~\cite{cosim-survey}. The identified features are
available in a rough feature model at \url{} and the identified use cases are available at \url{}.

It is an ongoing task to map use cases to features, and to map features to
language constructs and plugins.
As mentioned in the abstract, the Maestro team is very interested in features,
use cases and similar. Feedback can be provided via GitHub issues at the Maestro
repository \url{https://github.com/INTO-CPS-Association/maestrov2/issues} or via email to casper.thule@eng.au.dk.

\clearpage
\section{Approach to Conducting a Co-simulation}
\label{sec:approach}

\clearpage
\section{Maestro Base Language}
\label{sec:mabl}

\clearpage
\section{Plugins}
\label{sec:plugins}

\clearpage
\input{sections/examples.tex}
\clearpage
\input{sections/future_work.tex}
\clearpage
\input{sections/legacy_context.tex}
\clearpage
\input{sections/legacy_intro.tex}
\clearpage
%
%
%
%
%%%% Bibliography %%%%
\bibliographystyle{alpha}
\bibliography{bibliography}
\label{ch:bib} %label to refer to
%
%
%
\clearpage
%
%
%
\appendix
\section{List of Acronyms}\label{appendix:acronyms}
\begin{longtable}{ll}
DSL & Domain Specific Language\\
\end{longtable}

\clearpage
%
%
%
\end{document}

%%% Local Variables:
%%% mode: latex
%%% TeX-master: t
%%% End:
